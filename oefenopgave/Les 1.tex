Date : 7 februari 2024

wisselspanning motor:
\begin{itemize}
    \item asynchroon week in week uit draaien compact weinig onderhoud.
    \item synchroon nog zwaardere insudstrie energie opwekking.
    \item Stappen motor.
\end{itemize}

Toets:
\begin{itemize}
    \item Opdrachten over de 4 onderwerpen wordt individueel uitgevoerd 50%.
    \item Casus in tweetallen 50%.
\end{itemize}

\subsection{DC-motor}

Elektrisch:

elektrisch is input; spanning [V], stroom [I], vermogen [W]
mechanisch is output; torque [nm], rpm [omw/min], vermogen[W]

koppel = kracht * arm -> [Nm] = [N]*[m]

een dc motor kan je schematisch weergeven met een weerstand en een spoel.
impedantie spoel-> Zl = jwl is het verband tussen de spanning en de stroom.
Ul = Il * Zl
wl = 2 * pi * f * L

Ir = (U voeding - U motor) / 
Hoe harder de motor gaat draaien hoe meet spanning de motor opwekt en hoe minder stroom er gaat lopen

bij gelijkstoorm is de f 0 dus is de spoel een draadje

LEICIE = ezelsbruggetje -> bij een spoel spanning 90 graden voor op de stroom en bij een condensaotor andersom

mechanisch:

Umut= d*flux/dt=Kem * w * sinw * t

met commutatie wordt dit: Umut = Kem * W
Kem:[v.s/rad]
